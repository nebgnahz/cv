\documentclass[11pt, letterpaper]{awesome-cv}

\geometry{left=2cm, top=1.8cm, right=2cm, bottom=1.8cm, footskip=.8cm}
\fontdir[fonts/]
\colorlet{awesome}{awesome-skyblue}
\setbool{acvSectionColorHighlight}{true}
\renewcommand{\acvHeaderSocialSep}{\quad\textbar\quad}

\name{Ben}{Zhang}
\position{Software Engineer at Google}

\mobile{+1 (510) 982-6748}
\email{benzh@cs.berkeley.edu}
\homepage{www.benzhang.name}
\github{nebgnahz}

%%% Local Variables:
%%% mode: latex
%%% TeX-master: "cv"
%%% End:


\recipient
{Netflix Recruitment Team}
{100 Winchester Circle\\Los Gatos, CA 95032}

\letterdate{\today}
\letteropening{Dear Hiring Manager,}
\letterclosing{Sincerely,}
\letterenclosure[Attached]{Curriculum Vitae}

%-------------------------------------------------------------------------------
\begin{document}

\makecvheader[R]
\makecvfooter
  {\today}
  {Ben Zhang~~~·~~~Cover Letter}
  {}

\makelettertitle

%-------------------------------------------------------------------------------
%  LETTER CONTENT
%-------------------------------------------------------------------------------
\begin{cvletter}

  \lettersection{About Me}

  I am a Computer Science Ph.D. candidate at Berkeley and I am expected to
  graduate in Summer 2018. My research focuses on improving quality of service
  for mobile and Internet of Things (IoT) applications under resource
  constraints, such as unreliable network or limited computing resources. I am a
  system builder and I enjoy \textbf{building and optimizing systems
    iteratively}. I am looking for software engineering positions at Netflix
  focusing on infrastructure and back-end systems.

  \lettersection{Why Netflix?}

  Netflix is the leading company for video streaming. While delivering videos
  seems easy given today's technology and tools, doing it at Netflix scale---100
  million members watching over 140 million hours of content every day---is
  non-trivial. Behind the scene is strong engineering, such as performance
  tuning (AWS EC2, CPU, memory, networking, etc) and well-architected systems
  (e.g. Open Connect). Working at Netflix will give me the opportunity to design
  and implement large-scale distributed systems and see how it impacts millions
  of users.

  In addition, I am a happy Netflix customer. I have spent many hours
  binge-watching Netflix original shows (too long to list here). And I use
  \url{https://fast.com/} to measure available bandwidth. Netflix products offer
  a smooth user experience, from both user interfaces and functionalities. I
  would like to be a part of the teams that make these great products.

  \lettersection{Why Me?}

  My past research and industrial experience makes me a great fit for Netflix
  software engineer. $(i)$ In my research, to address the resource constrains
  for mobile and IoT applications, I have used adaptive techniques similar to
  adaptive video streaming. Because the uplink is usually the bottleneck for
  these applications, and different applications require different adaptation
  strategies, existing adaptive systems and tools do not work. I have built a
  framework that incorporates adaptation from scratch. $(ii)$ My internship
  experience also aligns well with Netflix. In my 2014 summer internship at
  Google, I optimized the wide-area crawling performance by scheduling requests
  based on geolocations, similar to how Netflix optimizes the delivery network.

  While my work experience is limited, at Berkeley, we build systems beyond
  prototype stages, and often in a large team. I am part of the TerraSwarm
  Research Center and have been actively involved or participated in several
  projects spanning research labs or even universities. We dogfood our own
  systems internally and release some tools (such as ESP and GDP, see my CV) for
  the public. The software engineering experience makes me different from normal
  fresh graduates and I have confidence to quickly adjust myself from the
  research environment to industry and start the software engineer job.

\end{cvletter}

%-------------------------------------------------------------------------------
% Print the signature and enclosures with above letter informations
\makeletterclosing

\end{document}

%%% Local Variables:
%%% coding: utf-8
%%% mode: latex
%%% TeX-engine: xetex
%%% TeX-master: t
%%% End: